\documentclass{article}

\usepackage[margin=1in]{geometry}
\usepackage{amsmath,amssymb}

\begin{document}

\title{Gradients}
\author{Otto and Joost}
\maketitle

\section{h}
\begin{align*}
	\frac{\partial \log p_\theta (x|z)}{\partial W_3} = \frac{\partial \log p_\theta}{\partial y} \frac{\partial y}{\partial z} \frac{\partial z}{\partial h} \frac{\partial h}{\partial W_3} \\
	\frac{\partial h}{\partial W_3} = x(1-\tanh^2(W_3x+b_3)) \\
	\frac{\partial z}{\partial h} = \frac{\partial \mu}{\partial h} + \epsilon \frac{\partial \sigma}{\partial h} \\
	\frac{\partial \mu}{\partial h} = W_4 \\
	\frac{\partial \sigma}{\partial h} = \frac{W_5 e^{W_5h + b_5}}{2 \sqrt{e^{W_5h + b_5}}}
	=  \frac{1}{2} W_5 \sqrt{e^{W_5h + b_5}} \\
\end{align*}

For y, we call what is after the minus sign in the exponent g.

\begin{align*}
	\frac{\partial y}{\partial z} &= \frac{\partial y}{\partial g} \frac{\partial g}{\partial z} \\
	\frac{\partial g}{\partial z} &= W_2(W_1(1- \tanh^2(W_1 z + b_1))) \exp(W_2 \tanh(W_1 z + b_1) + b_2) \\
	\frac{\partial y}{\partial g} &= y(g)(1 - y(g))
\end{align*}

\end{document}
